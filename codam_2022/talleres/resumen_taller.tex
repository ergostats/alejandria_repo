% Options for packages loaded elsewhere
\PassOptionsToPackage{unicode}{hyperref}
\PassOptionsToPackage{hyphens}{url}
%
\documentclass[
]{article}
\usepackage{amsmath,amssymb}
\usepackage{lmodern}
\usepackage{iftex}
\ifPDFTeX
  \usepackage[T1]{fontenc}
  \usepackage[utf8]{inputenc}
  \usepackage{textcomp} % provide euro and other symbols
\else % if luatex or xetex
  \usepackage{unicode-math}
  \defaultfontfeatures{Scale=MatchLowercase}
  \defaultfontfeatures[\rmfamily]{Ligatures=TeX,Scale=1}
\fi
% Use upquote if available, for straight quotes in verbatim environments
\IfFileExists{upquote.sty}{\usepackage{upquote}}{}
\IfFileExists{microtype.sty}{% use microtype if available
  \usepackage[]{microtype}
  \UseMicrotypeSet[protrusion]{basicmath} % disable protrusion for tt fonts
}{}
\makeatletter
\@ifundefined{KOMAClassName}{% if non-KOMA class
  \IfFileExists{parskip.sty}{%
    \usepackage{parskip}
  }{% else
    \setlength{\parindent}{0pt}
    \setlength{\parskip}{6pt plus 2pt minus 1pt}}
}{% if KOMA class
  \KOMAoptions{parskip=half}}
\makeatother
\usepackage{xcolor}
\IfFileExists{xurl.sty}{\usepackage{xurl}}{} % add URL line breaks if available
\IfFileExists{bookmark.sty}{\usepackage{bookmark}}{\usepackage{hyperref}}
\hypersetup{
  pdftitle={Investigación colaborativa con R},
  pdfauthor={Bajaña, Alex; Chanatasig Evelyn; Heredia, Aracely; Lombeida, Esteban},
  hidelinks,
  pdfcreator={LaTeX via pandoc}}
\urlstyle{same} % disable monospaced font for URLs
\usepackage[margin=1in]{geometry}
\usepackage{graphicx}
\makeatletter
\def\maxwidth{\ifdim\Gin@nat@width>\linewidth\linewidth\else\Gin@nat@width\fi}
\def\maxheight{\ifdim\Gin@nat@height>\textheight\textheight\else\Gin@nat@height\fi}
\makeatother
% Scale images if necessary, so that they will not overflow the page
% margins by default, and it is still possible to overwrite the defaults
% using explicit options in \includegraphics[width, height, ...]{}
\setkeys{Gin}{width=\maxwidth,height=\maxheight,keepaspectratio}
% Set default figure placement to htbp
\makeatletter
\def\fps@figure{htbp}
\makeatother
\setlength{\emergencystretch}{3em} % prevent overfull lines
\providecommand{\tightlist}{%
  \setlength{\itemsep}{0pt}\setlength{\parskip}{0pt}}
\setcounter{secnumdepth}{5}
\ifLuaTeX
  \usepackage{selnolig}  % disable illegal ligatures
\fi

\title{Investigación colaborativa con R}
\usepackage{etoolbox}
\makeatletter
\providecommand{\subtitle}[1]{% add subtitle to \maketitle
  \apptocmd{\@title}{\par {\large #1 \par}}{}{}
}
\makeatother
\subtitle{Tópico: Datos y enfoque en investigaciones inclusivas}
\author{Bajaña, Alex \and Chanatasig Evelyn \and Heredia,
Aracely \and Lombeida, Esteban}
\date{2022-05-29}

\begin{document}
\maketitle

{
\setcounter{tocdepth}{1}
\tableofcontents
}
\hypertarget{resumen}{%
\section{Resumen}\label{resumen}}

El desarrollo de una investigación académica, un reporte profesional o
un artículo periodístico, implica la recopilación de una serie de
recursos como son bases de datos, bibliografía especializada o análisis
personales previos. Recursos con los cuales pretendemos justificar la
importancia de nuestros objetivos y la forma de llevarlos a cabo.

Es claro que un sistema de registro de estos recursos no solo es
importante, sino que es la esencia misma de la reproducibilidad de
nuestra investigación o documento. Más aún cuando trabajamos de manera
colaborativa. De modo que cada miembro del equipo pueda llevar un
registro de sus recursos de manera accesible, comprensible y segura para
que pueda cumplir con su rol de manera efectiva.

La discusión de los resultados preliminares de una investigación
provienen de la integración de los múltiples esfuerzos de cada uno de
los miembros del equipo. Integración que, en caso de ser ordenada
generará no solo el mecanismo para la consecución de los objetivos
planteados, sino que servirá también de ejemplo para otras experiencias
de investigación reproducible en un ambiente colaborativo.

El sistema Alejandría está inspirado en la gran biblioteca del pasado
que hoy está cubierta de misticismo e historia. Así como los documentos
y objetos de distintas eras y culturas, Alejandría consta de un
repositorio para la administración de scripts y archivos de desarrollo
implementado en GitHub, un sistema de almacenamiento en OneDrive y, el
faro que ilumina nuestras investigaciones, el tablero Alejandría que
está implementado con la librería \{pins\} de R.

En el taller de 1 hora y 20 minutos, más 25 minutos de preguntas,
enseñaremos la implementación de un sistema como Alejandría para equipos
de investigación que buscan una forma efectiva de trabajo colaborativo
empleando herramientas de código abierto. Se requiere experiencia con R
y Rstudio para obtener mayor valor del taller.

\end{document}
